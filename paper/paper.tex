\title{Relating the diversity of interneuronal subtypes to their functional roles in development, homeostasis and sensory processing}
\author
{Philipp Rudiger,$^{1\ast}$ James A. Bednar,$^{1}$\\
\\
\normalsize{$^{1}$Institute for Adaptive and Neural Computation, University of Edinburgh,}\\
\normalsize{School of Informatics, Edinburgh, EH8 9AB, United Kingdom}\\
\\
\normalsize{$^\ast$To whom correspondence should be addressed; E-mail:  P.Rudiger@ed.ac.uk.}
}
\date{\today}

\documentclass[12pt]{article}

\usepackage{natbib}
\usepackage{gensymb}
\usepackage[margin=0.5in]{geometry}

\begin{document}
\maketitle

\begin{abstract}
A circuit-level understanding of the cortex has long been sought, but
only recently have improvements in genetic and imaging methods allowed
systematic investigation of the roles of different interneuron
classes. In particular, the chemical markers parvalbumin (PV),
somatostatin (Sst) and vasoactive-intestinal peptide (VIP) have been
found to divide interneurons into three non-overlapping subtypes that
roughly correspond to previous, functionally defined subtypes
\cite{Pfeffer2013}. Studies show roles for these cells in a wide array
of processes across different time scales, including development,
homeostasis, and instantaneous sensory responses. We present an
analysis investigating the involvement of these interneuron classes
across all of these spatial and temporal scales, using a rate-based
model of map development in primary visual cortex. The model includes
the main anatomical and functional properties of the two most common
GABAergic neuron types in the cortex, PV+ and Sst+ interneurons
\cite{Pfeffer2013}. PV+ neurons receive strong feedforward input,
provide perisomatic inhibition, and are typically only weakly feature
selective \cite{Hofer2011}, while Sst+ neurons preferentially synapse
on distal dendrites and mostly receive input from horizontal axons
within layer 2/3 \cite{Xu2009}. Sst+ neurons are driven by
facilitating synapses, and thus respond strongly only for high
contrasts or large stimuli \cite{Adesnik2012}, giving them lower
firing rates, stronger feature preference, and longer response
latencies. The model allows us to link particular properties of these
inhibitory neurons to specific functional roles, through parameter
space exploration and lesion studies. The results show that the fast
response of PV+ neurons makes them suited to balance feedforward
excitation, while their broad activation helps sparsify cortical
activity. This suggests a central role of PV+ neurons in map
development and maintaining stability in the early sensory
response. In contrast, Sst+ neurons are primarily recruited under high
contrast conditions or for large stimuli, which results in a
contrast-dependent switch in surround modulation from facilitation to
suppression. Overall, our analysis suggests that PV+ neurons are
central to organized development and balancing the instantaneous
response, while Sst+ neurons integrate over larger temporal and
spatial scales and mediate a range of surround modulation effects.
\end{abstract}

\section{Introduction}
\begin{table}
  \centering
  \begin{tabular}{l | l l l l}
    Connection               & Literature            & Species & Layer & $\sigma$ \\
    \hline
    LGN-V1 Afferents         & \cite{Angelucci2002c} & macaque & 4C$\alpha$ & $0.8-1.6\degree$ \\ 
                             & \cite{Angelucci2006a} & macaque & 4A/4C$\beta$ & $0.91 \pm 0.041 \degree$ \\
    \hline
    V1 local excitation      & \cite{Buzas2006}      & cat      & 2-4 single cell & $288 \mu m$ \\
                             & \cite{Buzas2006}      & cat      & 2-4 population  & $520 \mu m$ \\
    \hline
    V1 basket cells          & \cite{Buzas2001}      & cat      & 2-6 & $0.7-1.9 \degree$ \\
                             & \cite{Buzas2001}      & cat      & 2-6 & $0.76-2.6 mm$ \\
    \hline
    V1 long-range excitation & \cite{Angelucci2002}  & macaque  & 2/3 & $6\pm 0.7 mm$ (3-9) \\
                             &                       &          & 4B/4C$\alpha$ & $6.7 \pm 0.7 mm$ (4.7-10) \\
                             &                       &          & population & $2.47 \pm 0.3 \degree$ \\
                             & \cite{Buzas2006}      & cat      & 2/3 & $6 mm$ \\
    \hline
  \end{tabular}
  \caption[]%
          {Anatomical estimates of the spatial profiles of V1 connectivity.}
  \label{anatomicaltable}
\end{table}

\begin{table}
  \centering
  \begin{tabular}{l | l l l l}
    Measurement              & Literature            & Species & Layer & $\sigma$ \\
    \hline
    LGN Excitatory DoG fit   & \cite{Sceniak2006}    & cat     & 4A/4C$\alpha$/4C$\beta$ & $0.22$/$0.51$/$0.46\degree$ \\
    LGN Inhibitory DoG fit   & \cite{Sceniak2006}    & cat     & 4A/4C$\alpha$/4C$\beta$ & $0.5$/$0.62$/$0.51\degree$ \\
    \hline
    hsRF                     & \cite{Levitt2002}     & macaque & 2-6 & $1.0 \pm 0.2 \degree$ (0.3 - 2.2) \\
    \hline
    V1 Excitatory DoG fit    & \cite{Levitt2002}     & macaque & 2-6 & $0.9 \degree$ \\
                             & \cite{Sceniak2001}    & cat     & 2-6 & $1.0 \degree$ \\
                             & \cite{Cavanaugh2002}  & macaque & 2-6 & $1.4 \degree$ \\
                             & \cite{Solomon2004}    & macaque & not stated & $0.94 \degree$ \\
    \hline
    V1 Inhibitory DoG fit    & \cite{Levitt2002}     & macaque & 2-6 & $1.9 \degree$ \\
                             & \cite{Sceniak2001}    & cat     & 2-6 & $2.2 \degree$ \\
                             & \cite{Cavanaugh2002}  & macaque & 2-6 & $2.7 \degree$ \\
                             & \cite{Solomon2004}    & macaque & not stated & $2.97 \degree$ \\
    
    \hline
  \end{tabular}
  \caption[]%
          {Anatomical estimates of the spatial profiles of V1 connectivity.}
  \label{anatomicaltable}
\end{table}

\bibliographystyle{apalike}
\bibliography{paper}

\end{document}
