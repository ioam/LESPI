% Use only LaTeX2e, calling the article.cls class and 12-point type.

\documentclass[12pt]{article}

% Users of the {thebibliography} environment or BibTeX should use the
% scicite.sty package, downloadable from *Science* at
% www.sciencemag.org/about/authors/prep/TeX_help/ .
% This package should properly format in-text
% reference calls and reference-list numbers.

\usepackage{scicite}

% Use times if you have the font installed; otherwise, comment out the
% following line.

\usepackage{times}

% The preamble here sets up a lot of new/revised commands and
% environments.  It's annoying, but please do *not* try to strip these
% out into a separate .sty file (which could lead to the loss of some
% information when we convert the file to other formats).  Instead, keep
% them in the preamble of your main LaTeX source file.


% The following parameters seem to provide a reasonable page setup.

\topmargin 0.0cm
\oddsidemargin 0.2cm
\textwidth 16cm 
\textheight 21cm
\footskip 1.0cm


%The next command sets up an environment for the abstract to your paper.

\newenvironment{sciabstract}{%
\begin{quote} \bf}
{\end{quote}}


% If your reference list includes text notes as well as references,
% include the following line; otherwise, comment it out.

\renewcommand\refname{References and Notes}

% The following lines set up an environment for the last note in the
% reference list, which commonly includes acknowledgments of funding,
% help, etc.  It's intended for users of BibTeX or the {thebibliography}
% environment.  Users who are hand-coding their references at the end
% using a list environment such as {enumerate} can simply add another
% item at the end, and it will be numbered automatically.

\newcounter{lastnote}
\newenvironment{scilastnote}{%
\setcounter{lastnote}{\value{enumiv}}%
\addtocounter{lastnote}{+1}%
\begin{list}%
{\arabic{lastnote}.}
{\setlength{\leftmargin}{.22in}}
{\setlength{\labelsep}{.5em}}}
{\end{list}}


% Include your paper's title here

\title{Relating the diversity of interneuronal subtypes to their functional roles in development, homeostasis and sensory processing} 


% Place the author information here.  Please hand-code the contact
% information and notecalls; do *not* use \footnote commands.  Let the
% author contact information appear immediately below the author names
% as shown.  We would also prefer that you don't change the type-size
% settings shown here.

\author
{Philipp Rudiger,$^{1\ast}$ James A. Bednar,$^{1}$\\
\\
\normalsize{$^{1}$Institute for Adaptive and Neural Computation, University of Edinburgh,}\\
\normalsize{School of Informatics, Edinburgh, EH8 9AB, United Kingdom}\\
\\
\normalsize{$^\ast$To whom correspondence should be addressed; E-mail:  P.Rudiger@ed.ac.uk.}
}

% Include the date command, but leave its argument blank.

\date{}



%%%%%%%%%%%%%%%%% END OF PREAMBLE %%%%%%%%%%%%%%%%



\begin{document} 

% Double-space the manuscript.

\baselineskip24pt

% Make the title.

\maketitle 



% Place your abstract within the special {sciabstract} environment.

\begin{sciabstract}
A circuit-level understanding of the cortex has long been sought, but
only recently have improvements in genetic and imaging methods allowed
systematic investigation of the roles of different interneuron
classes. In particular, the chemical markers parvalbumin (PV),
somatostatin (Sst) and vasoactive-intestinal peptide (VIP) have been
found to divide interneurons into three non-overlapping subtypes that
roughly correspond to previous, functionally defined subtypes
\cite{Pfeffer2013}. Studies show roles for these cells in a wide
array of processes across different time scales, including
development, homeostasis, and instantaneous sensory responses. We
present an analysis investigating the involvement of these interneuron
classes across all of these spatial and temporal scales, using a
rate-based model of map development in primary visual cortex. The
model includes the main anatomical and functional properties of the
two most common GABAergic neuron types in the cortex, PV+ and Sst+
interneurons \cite{Pfeffer2013}. PV+ neurons receive strong
feedforward input, provide perisomatic inhibition, and are typically
only weakly feature selective \cite{Hofer2011}, while Sst+ neurons
preferentially synapse on distal dendrites and mostly receive input
from horizontal axons within layer 2/3 \cite{Xu2009}. Sst+ neurons
are driven by facilitating synapses, and thus respond strongly only
for high contrasts or large stimuli \cite{Adesnik2012}, giving them
lower firing rates, stronger feature preference, and longer response
latencies. The model allows us to link particular properties of these
inhibitory neurons to specific functional roles, through parameter
space exploration and lesion studies. The results show that the fast
response of PV+ neurons makes them suited to balance feedforward
excitation, while their broad activation helps sparsify cortical
activity. This suggests a central role of PV+ neurons in map
development and maintaining stability in the early sensory
response. In contrast, Sst+ neurons are primarily recruited under high
contrast or large stimuli, which results in a contrast-dependent
switch in surround modulation from facilitation to
suppression. Overall, our analysis suggests that PV+ neurons are
central to organized development and balancing the instantaneous
response, while Sst+ neurons integrate over larger temporal and
spatial scales and mediate a range of surround modulation effects.
\end{sciabstract}



% In setting up this template for *Science* papers, we've used both
% the \section* command and the \paragraph* command for topical
% divisions.  Which you use will of course depend on the type of paper
% you're writing.  Review Articles tend to have displayed headings, for
% which \section* is more appropriate; Research Articles, when they have
% formal topical divisions at all, tend to signal them with bold text
% that runs into the paragraph, for which \paragraph* is the right
% choice.  Either way, use the asterisk (*) modifier, as shown, to
% suppress numbering.

\section*{Introduction}


% Your references go at the end of the main text, and before the
% figures.  For this document we've used BibTeX, the .bib file
% scibib.bib, and the .bst file Science.bst.  The package scicite.sty
% was included to format the reference numbers according to *Science*
% style.


\bibliography{scifile}

\bibliographystyle{Science}

\clearpage




\end{document}
